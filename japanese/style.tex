\renewcommand{\labelenumi}{\theenumi. }
\renewcommand{\labelenumii}{(\theenumii) }
\setlength{\parindent}{1\zw}
\setlength{\parskip}{0pt}

\lstdefinelanguage{haskell}{
  morekeywords={data,type,newtype,instance,class,module,import,if,then,else,where,deriving},
  sensitive=true,
  morecomment=[l]{--},
  morestring=[b]"
}

\definecolor{dkgreen}{rgb}{0,0.6,0}
\definecolor{gray}{rgb}{0.5,0.5,0.5}
\definecolor{mauve}{rgb}{0.58,0,0.82}

\lstset{frame=tb,
  language=haskell,
  aboveskip=3mm,
  belowskip=3mm,
  showstringspaces=false,
  columns=flexible,
  basicstyle={\small\ttfamily},
  numbers=none,
  numberstyle=\tiny\color{gray},
  keywordstyle=\color{blue},
  commentstyle=\color{dkgreen},
  stringstyle=\color{mauve},
  frame=single,
  breaklines=true,
  breakatwhitespace=true
  tabsize=4
}

% ダッシュをつなげる
% cf. https://qiita.com/isari/items/1d0b60b76c7ef168e376
\usepackage{newunicodechar}
\makeatletter
\chardef\my@J@horizbar="2015% Unicodeの2015
\newunicodechar{―}{\x@my@dash}
\def\x@my@dash{\@ifnextchar―{%
  \my@J@horizbar\kern-.5\zw\my@J@horizbar\kern-.5\zw}{%
    \my@J@horizbar}}
% 次が―なら2回目のkernまでを、そうでないなら普通の―を出力
\makeatother
