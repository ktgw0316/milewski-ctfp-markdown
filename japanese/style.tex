\renewcommand{\labelenumi}{\theenumi. }
\renewcommand{\labelenumii}{(\theenumii) }
\setlength{\parindent}{1\zw}
\setlength{\parskip}{0pt}

% Codeblock style
\DefineVerbatimEnvironment{Highlighting}{Verbatim}{commandchars=\\\{\},fontsize=\small}
\mdfsetup{
   linecolor=gray!20,
   linewidth=2pt,
   topline=false,
   bottomline=false,
   rightline=false,
   skipabove=1ex,
   skipbelow=1ex,
 }
\surroundwithmdframed[linewidth=2pt]{Highlighting}

% ダッシュをつなげる
% cf. https://qiita.com/isari/items/1d0b60b76c7ef168e376
\usepackage{newunicodechar}
\makeatletter
\chardef\my@J@horizbar="2015% Unicodeの2015
\newunicodechar{―}{\x@my@dash}
\def\x@my@dash{\@ifnextchar―{%
  \my@J@horizbar\kern-.5\zw\my@J@horizbar\kern-.5\zw}{%
    \my@J@horizbar}}
% 次が―なら2回目のkernまでを、そうでないなら普通の―を出力
\makeatother
